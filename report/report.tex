%% USEFUL LINKS:
%% -------------
%%
%% - UiO LaTeX guides:          https://www.mn.uio.no/ifi/tjenester/it/hjelp/latex/
%% - Mathematics:               https://en.wikibooks.org/wiki/LaTeX/Mathematics
%% - Physics:                   https://ctan.uib.no/macros/latex/contrib/physics/physics.pdf
%% - Basics of Tikz:            https://en.wikibooks.org/wiki/LaTeX/PGF/Tikz
%% - All the colors!            https://en.wikibooks.org/wiki/LaTeX/Colors
%% - How to make tables:        https://en.wikibooks.org/wiki/LaTeX/Tables
%% - Code listing styles:       https://en.wikibooks.org/wiki/LaTeX/Source_Code_Listings
%% - \includegraphics           https://en.wikibooks.org/wiki/LaTeX/Importing_Graphics
%% - Learn more about figures:  https://en.wikibooks.org/wiki/LaTeX/Floats,_Figures_and_Captions
%% - Automagic bibliography:    https://en.wikibooks.org/wiki/LaTeX/Bibliography_Management  (this one is kinda difficult the first time)
%%
%%                              (This document is of class "revtex4-1", the REVTeX Guide explains how the class works)
%%   REVTeX Guide:              http://www.physics.csbsju.edu/370/papers/Journal_Style_Manuals/auguide4-1.pdf
%%
%% COMPILING THE .pdf FILE IN THE LINUX IN THE TERMINAL
%% ----------------------------------------------------
%%
%% [terminal]$ pdflatex report_example.tex
%%
%% Run the command twice, always.
%%
%% When using references, footnotes, etc. you should run the following chain of commands:
%%
%% [terminal]$ pdflatex report_example.tex
%% [terminal]$ bibtex report_example
%% [terminal]$ pdflatex report_example.tex
%% [terminal]$ pdflatex report_example.tex
%%
%% This series of commands can of course be gathered into a single-line command:
%% [terminal]$ pdflatex report_example.tex && bibtex report_example.aux && pdflatex report_example.tex && pdflatex report_example.tex
%%
%% ----------------------------------------------------


\documentclass[english,notitlepage,reprint,nofootinbib]{revtex4-2}  % defines the basic parameters of the document
% For preview: skriv i terminal: latexmk -pdf -pvc filnavn
% If you want a single-column, remove "reprint"

% Allows special characters (including æøå)
\usepackage[utf8]{inputenc}
% \usepackage[english]{babel}

%% Note that you may need to download some of these packages manually, it depends on your setup.
%% I recommend downloading TeXMaker, because it includes a large library of the most common packages.

\usepackage{physics,amssymb}  % mathematical symbols (physics imports amsmath)
\usepackage{amsmath}
\usepackage{graphicx}         % include graphics such as plots
\usepackage{xcolor}           % set colors
\usepackage{hyperref}         % automagic cross-referencing
\usepackage{listings}         % display code
\usepackage{subfigure}        % imports a lot of cool and useful figure commands
% \usepackage{float}
%\usepackage[section]{placeins}
\usepackage{algorithm}
\usepackage[noend]{algpseudocode}
\usepackage{subfigure}
\usepackage{tikz}
\usetikzlibrary{quantikz}
% defines the color of hyperref objects
% Blending two colors:  blue!80!black  =  80% blue and 20% black
\hypersetup{ % this is just my personal choice, feel free to change things
	colorlinks,
	linkcolor={red!50!black},
	citecolor={blue!50!black},
	urlcolor={blue!80!black}}


% ===========================================

%\addbibresource{refs.bib} % Entries are in the "refs.bib" file

\begin{document}
	
	\title{Solving the Schrödinger Equation Using the 2D Crank-Nicolson Method}  % self-explanatory
	\author{Eloi Martaillé Richard,
	\
	Christophe Kristian Blomsen,
	\
	Ola Mårem
	\&
	Jørgen Armann Glenndal
    }
	\date{\today}                             % self-explanatory
	\noaffiliation                            % ignore this, but keep it.
	
	%This is how we create an abstract section.
	\begin{abstract}
Github link: \href{https://github.com/christopheblomsen/fys4150_pro5}{$https://github.com/christopheblomsen/fys4150\_pro5$}
\end{abstract}
	\maketitle	
	
	
	% ===========================================
	\section{Introduction} \label{sec:introduction}
	% ===========================================


	% ===========================================
	\section{Theory} \label{sec:theory}
	% ===========================================
	
	
	
	
	% ===========================================
	\section{Methods}\label{sec:methods}
	% ===========================================
	The bare Schrödinger equation can be written as 

	\begin{equation}\label{eq:bare Schrodinger}
		i \frac{\partial u}{\partial t} = -\frac{\partial^2 u}{\partial x^2} - \frac{\partial^2 u}{\partial y^2} + v(x,y) u.
	\end{equation} 
	
	In order to solve equation \ref{eq:bare Schrodinger} numerically, we must discetize all the terms and approximate
	the derivatives. The discretization is done by introducing a grid of points in the
	$xy$ plane containing all the values of $x$ and $y$ we will use. The time is discretized by using points, with equal spacing, along the time axis.\\ \\
	The first order time derivative on the left hand side of equation \ref{eq:bare Schrodinger} is approximated using the forward Euler method given by
	\begin{equation}
		\frac{\partial u}{\partial t} \approx \frac{u^{n+1}-u^n}{\Delta t},
	\end{equation}

	where $u^n$, in our case, is the dimentionless wavefunction in two dimentions.
	The superscript denotes the time step of $u$ and $\Delta t$ is the length of the time
	step, i.e. the spacing between points on the time axis.	For the second order spatial derivatives
	on the right hand side of equation \ref{eq:bare Schrodinger}
	we use Taylor approximations of second order. In the $x$ direction,
	the double spatial derivative is then given by
	
	\begin{equation}\label{ex:double x}
		\frac{\partial^2 u_{i,j}}{\partial x^2} \approx \frac{u_{i+1,j}-2u_{i,j}+u_{i-1,j}}{h^2},
	\end{equation}
	where the subscript denotes the $x$ and $y$ spatial step and $h$ is the spatial step size.
	The equation for the $y$ direction is a matter of flipping the variables from
	$x \rightarrow y$ and $i \rightarrow j$.
	in equation \ref{ex:double x}. The approximations for the derivatives are used in
	the Crank-Nicolson approach, which, is given by
	
	\begin{equation}
		\frac{u^{n+1}-u^{n}}{\Delta t} = \frac{1}{2}    \left(     F^{n+1} + F^n     \right),
	\end{equation}

	where the $F^{n+1}$ and $F^n$ terms are the right hand side of equation \ref{eq:bare Schrodinger}
	evaluated at different time steps. The superscript denotes the time step of the given variable, and $\Delta t$ is the length of the time step,
	i.e. the spacing between points on the time axis.\\ \\

	We rewrite equation \ref{eq:bare Schrodinger} into 

	Rewriting the terms in equation \ref{eq:bare Schrodinger}
	according to the Crank-Nicolson approach we get 

	\begin{equation}
	\frac{u^n_{i,j}}{\Delta t} = 
	\end{equation}
	

	
	
	
	Writing out all the terms in equation \ref{eq:bare Schrodinger}
	according to the Crank-Nicolson approach, and keeping the terms at time step $n+1$ on the left hand side and terms 
	at time step $n$ on the right hand side we get 


	

	


	
	
	% ===========================================
	\section{Results}\label{sec:results}
	% ===========================================
	heihei
	- halla
	
		
	% ===========================================
	\section{Conclusion}\label{sec:conclusion}
	% ===========================================
	
	
	
	
	\onecolumngrid
	\section*{Refrences}
	%\bibliographystyle{apalike}
	\bibliography{refs}
	
	
\end{document}
