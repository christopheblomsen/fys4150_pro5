%% USEFUL LINKS:
%% -------------
%%
%% - UiO LaTeX guides:          https://www.mn.uio.no/ifi/tjenester/it/hjelp/latex/
%% - Mathematics:               https://en.wikibooks.org/wiki/LaTeX/Mathematics
%% - Physics:                   https://ctan.uib.no/macros/latex/contrib/physics/physics.pdf
%% - Basics of Tikz:            https://en.wikibooks.org/wiki/LaTeX/PGF/Tikz
%% - All the colors!            https://en.wikibooks.org/wiki/LaTeX/Colors
%% - How to make tables:        https://en.wikibooks.org/wiki/LaTeX/Tables
%% - Code listing styles:       https://en.wikibooks.org/wiki/LaTeX/Source_Code_Listings
%% - \includegraphics           https://en.wikibooks.org/wiki/LaTeX/Importing_Graphics
%% - Learn more about figures:  https://en.wikibooks.org/wiki/LaTeX/Floats,_Figures_and_Captions
%% - Automagic bibliography:    https://en.wikibooks.org/wiki/LaTeX/Bibliography_Management  (this one is kinda difficult the first time)
%%
%%                              (This document is of class "revtex4-1", the REVTeX Guide explains how the class works)
%%   REVTeX Guide:              http://www.physics.csbsju.edu/370/papers/Journal_Style_Manuals/auguide4-1.pdf
%%
%% COMPILING THE .pdf FILE IN THE LINUX IN THE TERMINAL
%% ----------------------------------------------------
%%
%% [terminal]$ pdflatex report_example.tex
%%
%% Run the command twice, always.
%%
%% When using references, footnotes, etc. you should run the following chain of commands:
%%
%% [terminal]$ pdflatex report_example.tex
%% [terminal]$ bibtex report_example
%% [terminal]$ pdflatex report_example.tex
%% [terminal]$ pdflatex report_example.tex
%%
%% This series of commands can of course be gathered into a single-line command:
%% [terminal]$ pdflatex report_example.tex && bibtex report_example.aux && pdflatex report_example.tex && pdflatex report_example.tex
%%
%% ----------------------------------------------------


\documentclass[english,notitlepage,reprint,nofootinbib]{revtex4-2}  % defines the basic parameters of the document
% For preview: skriv i terminal: latexmk -pdf -pvc filnavn
% If you want a single-column, remove "reprint"

% Allows special characters (including æøå)
\usepackage[utf8]{inputenc}
% \usepackage[english]{babel}

%% Note that you may need to download some of these packages manually, it depends on your setup.
%% I recommend downloading TeXMaker, because it includes a large library of the most common packages.

\usepackage{physics,amssymb}  % mathematical symbols (physics imports amsmath)
\usepackage{amsmath}
\usepackage{graphicx}         % include graphics such as plots
\usepackage{xcolor}           % set colors
\usepackage{hyperref}         % automagic cross-referencing
\usepackage{listings}         % display code
\usepackage{subfigure}        % imports a lot of cool and useful figure commands
% \usepackage{float}
%\usepackage[section]{placeins}
\usepackage{algorithm}
\usepackage[noend]{algpseudocode}
\usepackage{subfigure}
\usepackage{tikz}
\usetikzlibrary{quantikz}
% defines the color of hyperref objects
% Blending two colors:  blue!80!black  =  80% blue and 20% black
\hypersetup{ % this is just my personal choice, feel free to change things
	colorlinks,
	linkcolor={red!50!black},
	citecolor={blue!50!black},
	urlcolor={blue!80!black}}


% ===========================================

%\addbibresource{refs.bib} % Entries are in the "refs.bib" file

\begin{document}
	
	\title{\Huge{Solving the Schrödinger Equation Using the 2D Crank-Nicolson Method}}  % self-explanatory
	\author{Eloi Martaillé Richard,
	\
	Christophe Kristian Blomsen,
	\
	Ola Mårem
	\&
	Jørgen Armann Glenndal
    }
	\date{\today}                             % self-explanatory
	\noaffiliation                            % ignore this, but keep it.
	
	%This is how we create an abstract section.
	\begin{abstract}
This paper solves the 2D time-dependent Schrödinger equation numerically for a single particle
	interacting through the different slits. To solve this PDE, we will apply the
	Crank-Nicolson method. For convergence check, we verified that the probability of finding our particle inside the box remains 1 at all times. The total probability was conserved within an
	error margin of  $10^{-14}$. We studied the particle's distribution probability at a certain time and explored the detection probability through our three kinds of slit setup.
	The code can be found in a GitHub repo:		\href{https://github.com/christopheblomsen/fys4150_pro5}{https://github.com/christopheblomsen/fys4150\_pro5}, 
	\\and the resulting animation in a YouTube playlist: \\
	\href{https://www.youtube.com/watch?v=6SQEoxnW6Yk&list=PLkOjS3Q17MZaZeQ6vd_3mROJriEFFc2_S }{https://www.youtube.com/watch?v=6SQEoxnW6Yk$\&$list=PLkOjS3Q17MZaZeQ6vd\_3mROJriEFFc2\_S}

\end{abstract}
	\maketitle	
	
	
	% ===========================================
	\section{Introduction} \label{sec:introduction}
	% ===========================================

	Light has always been an intriguing phenomenon in physics. It began in ancient Greece when Democritus argued that all things in the universe are composed of indivisible
	sub-components, i.e., particles. At the beginning of the eleventh century Ibn al-Haytham
	wrote a physics book describing light no longer as a particle but as a wave, using a
	pinhole lens to reflect, and refract rays of light. This started a heated debate about the
	wave-particle duality of light. All physicists joined one of the two sides and tried to
	prove that light was either a wave or a particle. \\

	In 1801, Thomas Young developed the double-slit experiment from the Huygens-Fresnel principle
	resulting in the discovery of wave interference of light\cite{ThomasYoung}.


	This experiment did not solve the debate, but the wave property of light began to dominate
	scientific thinking. We will need the introduction of Quantum Mechanics to show that
	light can behave as a particle and as a wave, with the Schrödinger'equation being derived
	from the wave equation. \\

	In this report, we use the 2D time-dependent Schrodinger's equation to simulate a particle's
	motion in different slit methods, namely no slit, one, two or three slits. Since the equation is a PDE, similar to the heat equation, we will apply the Crank-Nicolson method.
Originally this finite difference method was used to solve the heat equation\cite{Crank1947APM}. \\

	In section \ref{sec:theory} we will overview the theoretical aspect of our problem by introducing
	the Schrödinger equation with some light quantum mechanical motion. Then in section \ref{sec:methods}
	we will introduce the numerical solution applied with the Crank-Nicolson method and the simulations. Furthermore, we briefly introduce all results obtained from the simulations in section
	\ref{sec:results}. Finally, we discuss our results in section \ref{sec:discussion}, and provide a summary
	in section \ref{sec:conclusion}.

	% ===========================================
	\section{Theory} \label{sec:theory}
	% ===========================================
	The time-dependent Schrödinger equation is defined as

	\begin{equation}
	i \hbar \frac{d}{d t}|\Psi\rangle=\hat{H}|\Psi\rangle \label{eq:schro_eq}
	\end{equation}

	where $\hat{H}$ is the Hamiltonian operator representing the total energy of the system
	and $|\Psi\rangle$ is the vector state in the Hilbert space. $|\Psi\rangle$ is a vector
	containing all possible information on the system at a specific time $t$. However, the actual physical meaning of $|\Psi\rangle$ remains an open question as it is impossible to
	physically observe $|\Psi\rangle$ \cite{griffiths:quantumn}. To obtain the information
	from $|\Psi\rangle$, we have to apply a certain operator on it, giving us a probability
	through the Born rule.\\


	In this paper, we will consider the case of a single, non-relativistic particle in two dimensions.
	We will also work in the position space to express our vector
	in terms of the orthonormal basis of the position space $|x_i\rangle$. This allows us
	, from the Born rule to calculate the probability of finding our particle between
	$x_i$ with the discretization of  $|\Psi\rangle$  at time $t_n$

	\begin{equation}
		P(x_i;t_n) = \langle x_i |\Psi\rangle = |\Psi^n_i|^2 =  \Psi^{n\dagger}_i\Psi^n_i
	\end{equation}

	with $\Psi^\dagger$ being the complex-conjugate and assuming normalization of $
	|\Psi\rangle$. \\

	By expressing our 2D Hamiltonian operator, we can then rewrite our eq. \ref{eq:schro_eq}


	\colorbox{red}{look ugly}

	\begin{align*}
		& \quad i \hbar \frac{\partial}{\partial t} \Psi(x, y, t)= \\
		& \left[-\frac{\hbar^{2}}{2 m}\left(\frac{\partial^{2}}{\partial x^{2}}+\frac{\partial^{2}}{\partial y^{2}}\right)+V(x, y, t)\right] \Psi(x, y, t) \tag{2} \label{eq:explicit_scho}
	\end{align*}


	To simplify our simulation, we will scale away all dimensions in \eqref{eq:explicit_scho}.
	
	% ===========================================
	\section{Methods}\label{sec:methods}
	% ===========================================
	\subsection{The Bare Schrödinger Equation}
	\noindent
	The bare Schrödinger equation is dimentionless and can be written as 

	\begin{equation}\label{eq:bare Schrodinger}
		\begin{split}
		i \frac{\partial u}{\partial t} &= -\frac{\partial^2 u}{\partial x^2} - \frac{\partial^2 u}{\partial y^2} + v(x,y) u\\
		\Rightarrow \ \ \frac{\partial u}{\partial t} &= i\left(\frac{\partial^2 u}{\partial x^2} + \frac{\partial^2 u}{\partial y^2}\right) - iv(x,y) u.
		\end{split}
	\end{equation} 
	
	\noindent
	In order to solve equation \ref{eq:bare Schrodinger} numerically, we must discetize all the terms and approximate
	the derivatives. The discretization is done by introducing a grid of points in the
	$xy$ plane containing all the values of $x$ and $y$ we will use. The time is discretized by using points, with equal spacing, along the time axis.\\ \\
	The first order time derivative on the left hand side of equation \ref{eq:bare Schrodinger} is approximated using the forward Euler method given by
	\begin{equation}
		\frac{\partial u^{n}}{\partial t} \approx \frac{u^{n+1}-u^n}{\Delta t},
	\end{equation}

	\noindent
	where $u^n$, in our case, is the dimensionless wave-function in two dimensions.
	The superscript denotes the time point of $u$ and $\Delta t$ is the length of the time
	step, i.e. the spacing between points on the time axis.	For the second order spatial derivatives
	on the right hand side of equation \ref{eq:bare Schrodinger}
	we use Taylor approximations of second order. In the $x$ direction,
	the double spatial derivative is then given by
	
	\begin{equation}\label{ex:double x}
		\frac{\partial^2 u_{i,j}}{\partial x^2} \approx \frac{u_{i+1,j}-2u_{i,j}+u_{i-1,j}}{h^2},
	\end{equation}
	where the subscript denotes the $x$ and $y$ spatial points and $h$ is the spatial step size.
	The equation for the $y$ direction is a matter of flipping the variables from
	$x \rightarrow y$ and $i \rightarrow j$.
	in equation \ref{ex:double x}.
	\subsection{The Crank-Nicolson Approach}
	
	\noindent
	The approximations for the derivatives are used in
	the Crank-Nicolson approach, which is given by
	
	\begin{equation}
		\frac{u^{n+1}-u^{n}}{\Delta t} = \frac{1}{2}    \left(     F^{n+1} + F^n     \right),
	\end{equation}

	\noindent
	where the $F^{n+1}$ and $F^n$ terms are the right hand side of equation \ref{eq:bare Schrodinger}
	evaluated at the specified time points. The superscript denotes the time point of the given variable, and $\Delta t$ is the length of the time step,
	i.e. the spacing between points on the time axis.\\ \\

	\noindent
	Rewriting equation \ref{eq:bare Schrodinger}
	according to the Crank-Nicolson approach we get 
	\begin{equation}\label{eq:algo}
		\begin{aligned}
		\frac{u^{n+1}_{i,j}-u^{n}_{i,j}}{\Delta t} &=  \frac{1}{2}\Big(\left(i\left(\frac{\partial^2 u_{i,j}}{\partial x^2} + \frac{\partial^2 u_{i,j}}{\partial y^2}\right) -i v_{i,j} u_{i,j}\right)^{n+1} \\
	    &+   \left(i(\frac{\partial^2 u_{i,j}}{\partial x^2} + \frac{\partial^2 u_{i,j}}{\partial y^2}) -i v_{i,j} u_{i,j}\right)^{n} \Big)\\
	    &\ \\
		&=  \frac{1}{2}\Big(    \frac{i}{h^2}\big(  u^{n+1}_{i+1,j}-2u^{n+1}_{i,j}+u^{n+1}_{i-1,j}   +u^{n+1}_{i,j+1}\\
		&-2u^{n+1}_{i,j}+u^{n+1}_{i,j-1}\big)-iv_{i,j}u^{n+1}_{i,j}\\
		&+ \frac{i}{h^2}\big(  u^{n}_{i+1,j}-2u^{n}_{i,j}+u^{n}_{i-1,j}   +u^{n}_{i,j+1}\\
		&-2u^{n}_{i,j}+u^{n}_{i,j-1}\big)-iv_{i,j}u^{n}_{i,j}      \Big)\\ 
		&\ \\
		&\Rightarrow \\ 
		&\ \\
		&u^{n+1}_{i,j} -r\Big(  u^{n+1}_{i+1,j}-2u^{n+1}_{i,j}+u^{n+1}_{i-1,j} \\
		&+u^{n+1}_{i,j+1}-2u^{n+1}_{i,j}+u^{n+1}_{i,j-1}\Big) +\frac{i\Delta t}{2}v_{i,j}u^{n+1}_{i,j}\\
		&\ \\
		&= u^n_{i,j}+ r\Big(  u^{n}_{i+1,j}-2u^{n}_{i,j}+u^{n}_{i-1,j}\\
		&+u^{n}_{i,j+1}-2u^{n}_{i,j}+u^{n}_{i,j-1}\Big)-\frac{i\Delta t}{2}v_{i,j}u^{n}_{i,j},
		\end{aligned}
	\end{equation}
	\noindent
	where we have used that $r = \frac{i\Delta t}{2h^2}$. The terms at time
	point $n+1$ are on the left hand side and terms at time point $n$ are on the right
	hand side.\\
	

	\subsection{The Simulated System}
	\noindent
	We concider a simulation box where $x \in [0,1]$, $y \in [0,1]$ and $t \in [0,T]$. The $x$ and $y$ axes are discetized such
	that there are $M-1$ points along each axis including the boundaries. The letter $i$ denotes the
	spatial point on the $x$ axis and the letter $j$ denotes the spatial point on the $y$ axis.
	We use the Dirichlet boundary conditions in the $xy$ plane, i.e.
	\begin{itemize}
		\item $u(x=0, y, t) = 0$
		\item $u(x=1, y, t) = 0$
		\item $u(x, y=0, t) = 0$ 
		\item $u(x, y=1, t) = 0$,
	\end{itemize}
	so we do not need to calculate the boundaries for each time point. We do, however, need 
	to take the boundaries into concideration when calculating the inner points of the $xy$ grid.
	The inner points of the $xy$ grid are given by the indices $i \in [1,M-2]$ and 
	$j \in [1,M-2]$. Since we want to solve the system for $u{n+1}$ for all the inner
	spatial points, there will be one equation associated with every combination of $i$ and $j$. 
	Since there are $(M-2)^2$ combinations of $i$ and $j$ there will be 
	$(M-2)^2$ equations for each time point. We will therefore need to solve
	$(M-2)^2$ equations to get $u$ at the next time step. 

	By organizing all the combinations of $u_{i,j}$ into one column vector, 
	we see that we can rewrite the result of equation \ref{eq:algo} into
	
	
	%the column vector with the corresponding equations given by equation \ref{eq:algo} can be 
	%turned into 
	
	
	a matrix equation on the form 
	\begin{equation}
		Au^{n+1} = Bu^n.	
	\end{equation} 	
	where $A$ and $B$ are matrices. The $A$ matrix contains all the coefficients of all the equations
	for the different $u_{i,j}$'s at time point $n+1$. The $B$ matrix is the same matrix for the 
	$u_{i,j}$'s at time point $n$. Since $A$ and $B$ are time invariant matrices,
	we only need an initial condition of $u$ to solve the matrix equation and evolve the system forward in time.\\ \\
	
	
	
	\noindent
	For the initial condition of the dimensionless wave-function in two dimensions,
	i.e. $u^{0}$, we use an un-normalized Gaussian wave packet given by

	\begin{equation}\label{eq:wave packet}
	\begin{split}
		u(x,y,t=0) = \exp\Big(&-\frac{(x-x_c)^2}{2 \sigma_x^2} - \frac{(y-y_c)^2}{2 \sigma_y^2} \\
		&+ i p_x (x-x_c) + i p_y (y-y_c)\Big). \\
	\end{split}
	\end{equation}
	
	\noindent
	In equation \ref{eq:wave packet}, $x_c$ and $y_c$ are the central
	coordinates of the wave packet, $\sigma_x$ and $\sigma_y$ are the standard
	deviations in the $x$ and $y$ directions respectively and $p_x$ and $p_y$
	are the momenta of the wave packet.\\
	
	\noindent
	The potential, $V$, we use to simulate the slits is zero everywhere except 
	for the locations where the barrier is located. The potential in the
	barrier will be constant and given by $v_0$.    


	
	
	% ===========================================
	\section{Results}\label{sec:results}
	% ===========================================
	We now present the results for the open box, and the two-slit system, using the initial conditions in table \ref{tab:1}. We have used the potential in figure \ref{fig:potential} for the double slit.
	
	
	\begin{figure}[H]
		\centering
		\includegraphics[trim={2cm 0cm 2cm 1cm},clip,width=0.49\textwidth]{figures/double_slit.pdf}
		\caption{The potential used for the double slit with value $v_0$ from table \ref{tab:1}.}
		\label{fig:potential}
	\end{figure}
	
	\begin{table}[h!]
	\centering
	\caption{system parameters used for calculating the probability distribution for the no-slit and double slit.}
	\label{tab:1}
	\begin{tabular}{c | c | c} %Frode uses vertical lines, therefore, we can do it
	 Variable& No slit & Double slit\\
	\hline
	$h$                & $5\cross10^{-3}$                &\quad $5\cross10^{-3}$   \\
	$\Delta t$       & $2.5\cross10^{-5}$            &\quad $2.5\cross10^{-5}$  \\
	$T$                & $8\cross10^{-3}$               &\quad $8\cross10^{-3}$     \\
	$x_c$             & $2.5\cross10^{-1}$            &\quad $2.5\cross10^{-1}$  \\
	$\sigma_x$    & $5\cross10^{-2}$               &\quad $5\cross10^{-2}$     \\
	$p_x$             & $2\cross10^{2}$                &\quad $2\cross10^{2}$      \\
	$y_c$             & $5\cross10^{-1}$               &\quad $5\cross10^{-1}$     \\
	$\sigma_y$    & $5\cross10^{-2}$               &\quad $1\cross10^{-1}$     \\
	$p_y$             & $0$                                    &\quad $0$                         \\
	$v_0$             & $0$                                    &\quad $1\cross10^{10}$    \\
	Wall thickness      & $2\cross10^{-2}$               & $2\cross10^{-2}$        \\
	Slit aperture          & $0$                                     & $5\cross10{-2}$        \\
	Wall between slits & $5\cross10^{-2}$          & $5\cross10^{-2}$            \\
	\hline
	\end{tabular}
	\end{table}
	The resulting absolute error of these systems can, as functions of time, be seen in figure \ref{fig:prob7_error}.

	\begin{figure}[H]
		\centering
		\includegraphics[scale=0.55]{figures/problem7_error.pdf}
		\caption{Absolute error versus time, using the system parameters in table \ref{tab:1}.}
		\label{fig:prob7_error}
	\end{figure}
	We then find how the wave packet probability propagates through the two-slit system. We created a playlist of animations for the propagation of the probability function though time,  but we included snapshots from the animations here, first at time $t=0$.

	\begin{figure}[H]
		\centering
		\includegraphics[width=0.49\textwidth]{figures/problem8_P_0.000.pdf}
		\caption{Probability function at time $t=0$.}
		\label{fig:prob_P0}
	\end{figure}
	Here, $x$ and $y$ are defined as dimensionless position, $x, y\in[0, 1]$. We have then the system at time $t = 10^{-3}$ in figure \ref{fig:prob8_P1}. The particles position is observed to have a probability on both sides of the slits, as well as the origin of a interference pattern.

	\begin{figure}[H]
		\centering
		\includegraphics[width=0.49\textwidth]{figures/problem8_P_0.001.pdf}
		\caption{Probability function at time $t=10^{-3}$.}
		\label{fig:prob8_P1}
	\end{figure}
	The simulation is run until time $t=2\cross10^{-3}$, which can be seen in figure \ref{fig:prob8_P2}.

	\begin{figure}[H]
		\centering
		\includegraphics[width=0.49\textwidth]{figures/problem8_P_0.002.pdf}
		\caption{Probability function at time $t=2\cross10^{-3}$.}
		\label{fig:prob8_P2}
	\end{figure}
	We observe the probability bounces back from the "infinite" potentials of the walls. Next up we will look at how the real and imaginary parts of the Schrödinger equation evolves. The real part of the dimensionless wave function $u$, and how it evolves with time $t \in[0, 2\cross10^{-3}]$ can be seen in figure \ref{fig:prob8_Re0}, \ref{fig:prob8_Re1} and \ref{fig:prob8_Re2}.

	\begin{figure}[H]
		\centering
		\includegraphics[width=0.49\textwidth]{figures/problem8_U_Re_0.000.pdf}
		\caption{The real part of the wave function $u$ at time $t = 0$.}
		\label{fig:prob8_Re0}
	\end{figure}

	\begin{figure}[h!]
		\centering
		\includegraphics[width=0.49\textwidth]{figures/problem8_U_Re_0.001.pdf}
		\caption{The real part of the wave function $u$ at time $t = 10^{-3}$.}
		\label{fig:prob8_Re1}
	\end{figure}

	\begin{figure}[h!]
		\centering
		\includegraphics[width=0.49\textwidth]{figures/problem8_U_Re_0.002.pdf}
		\caption{The real part of the wave function $u$ at time $t = 2\cross10^{-3}$.}
		\label{fig:prob8_Re2}
	\end{figure}
	We then look at the imaginary part at equal times.
	\begin{figure}[h!]
		\centering
		\includegraphics[width=0.49\textwidth]{figures/problem8_U_Im_0.000.pdf}
		\caption{The imaginary part of the wave function $u$ at time $t = 0$.}
		\label{fig:prob8_Im0}
	\end{figure}
	
	\begin{figure}[h!]
		\centering
		\includegraphics[width=0.49\textwidth]{figures/problem8_U_Im_0.001.pdf}
		\caption{The imaginary part of the wave function $u$ at time $t = 10^{-3}$.}
		\label{fig:prob8_Im1}
	\end{figure}
	
	\begin{figure}[H]
		\centering
		\includegraphics[width=0.49\textwidth]{figures/problem8_U_Im_0.002.pdf}
		\caption{The imaginary part of the wave function $u$ at time $t = 2\cross10^{-3}$.}
		\label{fig:prob8_Im2}
	\end{figure}

	We now look at the  interference patterns caused by similar potentials to that in fig \ref{fig:potential} for single (figure \ref{fig:prob9_single}), double (figure \ref{fig:prob9_double}) and triple (figure \ref{fig:prob9_triple}) slit systems.
	\begin{figure}[h!]
		\centering
		\includegraphics[scale=0.55]{figures/problem9_single_slit.pdf}
		\caption{Normalized probability distribution pattern using the single slit system.}
		\label{fig:prob9_single}
	\end{figure}
	
	\begin{figure}[h!]
		\centering
		\includegraphics[scale=0.55]{figures/problem9_double_slit.pdf}
		\caption{Normalized probability distribution using the double slit system.}
		\label{fig:prob9_double}
	\end{figure}
	
	\begin{figure}[H]
		\centering
		\includegraphics[scale=0.55]{figures/problem9_triple_slit.pdf}
		\caption{Normalized probability distribution using the triple slit system.}
		\label{fig:prob9_triple}
	\end{figure}

	% ===========================================
	\section{Discussion}\label{sec:discussion}
	% ===========================================
\subsection{Total probability deviation} \label{subsec:tot_prob_dev}

	One important aspect of numerical analysis is ensuring our numerical
	solution does not diverge. It is crucial, especially when no
	analytical solutions exist or are known. For example, in the case of an
	Newtonian system of equations, we usually have energy conservation,
	so checking that our energy is conserved is a good indicator that our
	results are significant. In the case of quantum mechanics, since we
	can only observe probabilities, one good convergence check is to ensure
	that our total probability is 1 at all times during our simulation. \\

	In figure \ref{fig:prob7_error}, we have plotted the absolute error of
	the total probability $|P_{\rm tot} - 1|$ for all time steps.
	Our total probability is always close to the theoretical
	value of 1 with an error margin of $10^{-14}$.  \\

	We made a function in our python script to make sure that all further
	simulations also pass our benchmark.

	\subsection{Color-map of the wave function and probability distribution}\label{subsec:colormap}

	Now we have simulated the parameters for the double-slit found in
	table \ref{tab:1} and we have plotted the real, imaginary, and probability
	distribution for times $t \in \{0, 0.001, 0.002 \}$s. \\

	At $t=0$ figures \ref{fig:prob8_Im0}, \ref{fig:prob8_Re0} and \ref{fig:prob_P0} shows the imaginary, real part and the distribution probability
	of our initial Gaussian wave-packet. We also notice that the real and imaginary
	parts have opposite values, making the complex norm look like our expected probability distribution. \\

	At $t=0.001$ Figures \ref{fig:prob8_Im1}, \ref{fig:prob8_Re1} and \ref{fig:prob8_P1}
	represent our particle hitting the double-slit configuration. We first see that our
	particle only moved in the $x$-direction, which is as expected because its the direction where momentum were initiated. When our particle hit the slit configuration, we can see that it behaves like a wave, with some parts being reflected with negative and positive values due to the constructive or destructive interference pattern. On the other side, we can see that a part of the wave went through the slit but has not shown self-interaction yet. \\


	The plot for our final time stamp $t=0.002$ are represented by
	figures \ref{fig:prob8_Im2}, \ref{fig:prob8_Re2} and \ref{fig:prob8_P2}.
	this time shows the wave function after it has passed the slit configuration.
	We now observe interference patterns in both the reflected and transmitted parts. This is expected, as a wave will interact with itself. We can see that the particle has a greater probability of being reflected than moving through due to the size difference
	between the slit aperture and the wall. We can observe two spots of high probability in
	the reflected part ($x\approx0.3$, $y\approx0.60$ and $y\approx0.30$). This is due to constructive interference.

	We see three main regions on the transmitted part, one more probable than the others. The wave spreading creates an interference pattern, meaning we observe minima and maxima. \\

	The real and imaginary parts of the wave function can not be described physically as it is impossible to observe. We can only give some sense when we put them
	together by calculating the complex norm giving us the probability distribution. We can
	see that all imaginary and real parts have a similar behaviour to our wave.\\

	\subsection{Detection Probability} \label{subsec:detec_prob}

	One last interesting thing we look at is the detection probability of our particle.
	To simulate this, we imagined a detector screen at $x=0.8$ and took the probability
	distribution along that line. We assume that we detect the particle, so we normalized our line distribution so that the sum of all probability goes to 1. We used this method
	for our three slit configuration plotted in Figures \ref{fig:prob9_single}, \ref{fig:prob9_double}
	and \ref{fig:prob9_triple}. \\

	For the single slit, we see a Gaussian shape-like profile around the centre where the slit
	is. In that case, the wave interacts with itself, but no interference is present.	For the
	double slit, we see one maximum at the centre and two secondary maxima on the edges. We now observe some interference patterns due to slit configuration. Making the wave interfere constructively on the maxima and destructively in between the pics. Finally, for
	the triple slit, we observe an interference pattern with four maxima and three
	minima. However, we do not observe the maxima around the centre. One interesting thing would be to use the classical wave equation, solve the slit configuration, and compare it with our results.
	
	% ===========================================
	\section{Conclusion}\label{sec:conclusion}
	% ===========================================
	
	In this paper we simulated a single non-relativistic particle obeying 
	Schrödinger's equation through three types of slit setup, single, double
	and triple slit. We observed that most of the wave-
	function was reflected while a small part passed through when we included slits. We observed the 
	interference pattern, similar to that for a wave. The detection probability if we would have placed 
	a detector screen behind the wall at $x=0.8$ was found, and the maxima and 
	minima of the detection probability due to constructive or destructive interference was observed. All simulation passed the convergence check by conserving the total 
	probability value with a negligible error of order  $10^{-14}$. \\
	
	
	
	\onecolumngrid
	\section*{References}
	%\bibliographystyle{apalike}
	\bibliography{refs}
	
	
\end{document}
