%% USEFUL LINKS:
%% -------------
%%
%% - UiO LaTeX guides:          https://www.mn.uio.no/ifi/tjenester/it/hjelp/latex/
%% - Mathematics:               https://en.wikibooks.org/wiki/LaTeX/Mathematics
%% - Physics:                   https://ctan.uib.no/macros/latex/contrib/physics/physics.pdf
%% - Basics of Tikz:            https://en.wikibooks.org/wiki/LaTeX/PGF/Tikz
%% - All the colors!            https://en.wikibooks.org/wiki/LaTeX/Colors
%% - How to make tables:        https://en.wikibooks.org/wiki/LaTeX/Tables
%% - Code listing styles:       https://en.wikibooks.org/wiki/LaTeX/Source_Code_Listings
%% - \includegraphics           https://en.wikibooks.org/wiki/LaTeX/Importing_Graphics
%% - Learn more about figures:  https://en.wikibooks.org/wiki/LaTeX/Floats,_Figures_and_Captions
%% - Automagic bibliography:    https://en.wikibooks.org/wiki/LaTeX/Bibliography_Management  (this one is kinda difficult the first time)
%%
%%                              (This document is of class "revtex4-1", the REVTeX Guide explains how the class works)
%%   REVTeX Guide:              http://www.physics.csbsju.edu/370/papers/Journal_Style_Manuals/auguide4-1.pdf
%%
%% COMPILING THE .pdf FILE IN THE LINUX IN THE TERMINAL
%% ----------------------------------------------------
%%
%% [terminal]$ pdflatex report_example.tex
%%
%% Run the command twice, always.
%%
%% When using references, footnotes, etc. you should run the following chain of commands:
%%
%% [terminal]$ pdflatex report_example.tex
%% [terminal]$ bibtex report_example
%% [terminal]$ pdflatex report_example.tex
%% [terminal]$ pdflatex report_example.tex
%%
%% This series of commands can of course be gathered into a single-line command:
%% [terminal]$ pdflatex report_example.tex && bibtex report_example.aux && pdflatex report_example.tex && pdflatex report_example.tex
%%
%% ----------------------------------------------------


\documentclass[english,notitlepage,reprint,nofootinbib]{revtex4-2}  % defines the basic parameters of the document
% For preview: skriv i terminal: latexmk -pdf -pvc filnavn
% If you want a single-column, remove "reprint"

% Allows special characters (including æøå)
\usepackage[utf8]{inputenc}
% \usepackage[english]{babel}

%% Note that you may need to download some of these packages manually, it depends on your setup.
%% I recommend downloading TeXMaker, because it includes a large library of the most common packages.

\usepackage{physics,amssymb}  % mathematical symbols (physics imports amsmath)
\usepackage{amsmath}
\usepackage{graphicx}         % include graphics such as plots
\usepackage{xcolor}           % set colors
\usepackage{hyperref}         % automagic cross-referencing
\usepackage{listings}         % display code
\usepackage{subfigure}        % imports a lot of cool and useful figure commands
% \usepackage{float}
%\usepackage[section]{placeins}
\usepackage{algorithm}
\usepackage[noend]{algpseudocode}
\usepackage{subfigure}
\usepackage{tikz}
\usetikzlibrary{quantikz}
% defines the color of hyperref objects
% Blending two colors:  blue!80!black  =  80% blue and 20% black
\hypersetup{ % this is just my personal choice, feel free to change things
	colorlinks,
	linkcolor={red!50!black},
	citecolor={blue!50!black},
	urlcolor={blue!80!black}}


% ===========================================

%\addbibresource{refs.bib} % Entries are in the "refs.bib" file

\begin{document}
	
	\title{Solving the Schrödinger Equation Using the 2D Crank-Nicolson Method}  % self-explanatory
	\author{Eloi Martaillé Richard,
	\
	Christophe Kristian Blomsen,
	\
	Ola Mårem
	\&
	Jørgen Armann Glenndal
    }
	\date{\today}                             % self-explanatory
	\noaffiliation                            % ignore this, but keep it.
	
	%This is how we create an abstract section.
	\begin{abstract}
This paper solves the 2D time-dependent Schrödinger equation numerically for a single particle
	interacting through the different slits. To solve this PDE, we will apply the
	Crank-Nicolson method. For convergence check, we verified that the probability of finding our particle inside the box remains 1 at all times. The total probability was conserved within an
	error margin of  $10^{-14}$. We studied the particle's distribution probability at a certain time and explored the detection probability through our three kinds of slit setup.
	The code can be found in a GitHub repo:		\href{https://github.com/christopheblomsen/fys4150_pro5}{https://github.com/christopheblomsen/fys4150\_pro5}

\end{abstract}
	\maketitle	
	
	
	% ===========================================
	\section{Introduction} \label{sec:introduction}
	% ===========================================

	Light has always been an intriguing phenomenon in physics. It began in ancient Greece when Democritus argued that all things in the universe are composed of indivisible
	sub-components, i.e., particles. At the beginning of the eleventh century Ibn al-Haytham
	wrote a physics book describing light no longer as a particle but as a wave, using a
	pinhole lens to reflect, and refract rays of light. This started a heated debate about the
	wave-particle duality of light. All physicists joined one of the two sides and tried to
	prove that light was either a wave or a particle. \\

	In 1801, Thomas Young developed the double-slit experiment from the Huygens-Fresnel principle
	resulting in the discovery of wave interference of light\cite{ThomasYoung}.


	This experiment did not solve the debate, but the wave property of light began to dominate
	scientific thinking. We will need the introduction of Quantum Mechanics to show that
	light can behave as a particle and as a wave, with the Schrödinger'equation being derived
	from the wave equation. \\

	In this report, we use the 2D time-dependent Schrodinger's equation to simulate a particle's
	motion in different slit methods, namely no slit, one, two or three slits. Since the equation is a PDE, similar to the heat equation, we will apply the Crank-Nicolson method.
Originally this finite difference method was used to solve the heat equation\cite{Crank1947APM}. \\

	In section \ref{sec:theory} we will overview the theoretical aspect of our problem by introducing
	the Schrödinger equation with some light quantum mechanical motion. Then in section \ref{sec:methods}
	we will introduce the numerical solution applied with the Crank-Nicolson method and the simulation. After that, we will briefly introduce all results obtained from the simulation in the section
	\ref{sec:results}. Finally, we will discuss those results in section \ref{sec:discussion}, and
	in section  \ref{sec:conclusion}, we provide a summary.

	% ===========================================
	\section{Theory} \label{sec:theory}
	% ===========================================
	The time-dependent Schrödinger equation is defined as

	\begin{equation}
	i \hbar \frac{d}{d t}|\Psi\rangle=\hat{H}|\Psi\rangle \label{eq:schro_eq}
	\end{equation}

	where $\hat{H}$ is the Hamiltonian operator representing the total energy of the system
	and $|\Psi\rangle$ is the vector state in the Hilbert space. $|\Psi\rangle$ is a vector
	containing all possible information on the system at a specific time $t$. However, the actual physical meaning of $|\Psi\rangle$ remains an open question as it is impossible to
	physically observe $|\Psi\rangle$ \cite{griffiths:quantumn}. To obtain the information
	from $|\Psi\rangle$, we have to apply a certain operator on it, giving us a probability
	through the Born rule.\\


	In this paper, we will consider the case of a single, non-relativistic particle in two dimensions.
	We will also work in the position space to express our vector
	in terms of the orthonormal basis of the position space $|x_i\rangle$. This allows us
	, from the Born rule to calculate the probability of finding our particle between
	$x_i$ with the discretization of  $|\Psi\rangle$  at time $t_n$

	\begin{equation}
		P(x_i;t_n) = \langle x_i |\Psi\rangle = |\Psi^n_i|^2 =  \Psi^{n\dagger}_i\Psi^n_i
	\end{equation}

	with $\Psi^\dagger$ being the complex-conjugate and assuming normalization of $
	|\Psi\rangle$. \\

	By expressing our 2D Hamiltonian operator, we can then rewrite our eq. \ref{eq:schro_eq}


	\colorbox{red}{look ugly}

	\begin{align*}
		& \quad i \hbar \frac{\partial}{\partial t} \Psi(x, y, t)= \\
		& \left[-\frac{\hbar^{2}}{2 m}\left(\frac{\partial^{2}}{\partial x^{2}}+\frac{\partial^{2}}{\partial y^{2}}\right)+V(x, y, t)\right] \Psi(x, y, t) \tag{2} \label{eq:explicit_scho}
	\end{align*}


	To simplify our simulation, we will scale away all dimensions in \eqref{eq:explicit_scho}.
	
	% ===========================================
	\section{Methods}\label{sec:methods}
	% ===========================================
	\subsection{The Bare Schrödinger Equation}
	The bare Schrödinger equation can be written as 

	\begin{equation}\label{eq:bare Schrodinger}
		\begin{split}
		i \frac{\partial u}{\partial t} &= -\frac{\partial^2 u}{\partial x^2} - \frac{\partial^2 u}{\partial y^2} + v(x,y) u\\
		\Rightarrow \ \ \frac{\partial u}{\partial t} &= i\left(\frac{\partial^2 u}{\partial x^2} + \frac{\partial^2 u}{\partial y^2}\right) - iv(x,y) u.
		\end{split}
	\end{equation} 
	
	\noindent
	In order to solve equation \ref{eq:bare Schrodinger} numerically, we must discetize all the terms and approximate
	the derivatives. The discretization is done by introducing a grid of points in the
	$xy$ plane containing all the values of $x$ and $y$ we will use. The time is discretized by using points, with equal spacing, along the time axis.\\ \\
	The first order time derivative on the left hand side of equation \ref{eq:bare Schrodinger} is approximated using the forward Euler method given by
	\begin{equation}
		\frac{\partial u^{n}}{\partial t} \approx \frac{u^{n+1}-u^n}{\Delta t},
	\end{equation}

	\noindent
	where $u^n$, in our case, is the dimensionless wave-function in two dimensions.
	The superscript denotes the time step of $u$ and $\Delta t$ is the length of the time
	step, i.e. the spacing between points on the time axis.	For the second order spatial derivatives
	on the right hand side of equation \ref{eq:bare Schrodinger}
	we use Taylor approximations of second order. In the $x$ direction,
	the double spatial derivative is then given by
	
	\begin{equation}\label{ex:double x}
		\frac{\partial^2 u_{i,j}}{\partial x^2} \approx \frac{u_{i+1,j}-2u_{i,j}+u_{i-1,j}}{h^2},
	\end{equation}
	where the subscript denotes the $x$ and $y$ spatial step and $h$ is the spatial step size.
	The equation for the $y$ direction is a matter of flipping the variables from
	$x \rightarrow y$ and $i \rightarrow j$.
	in equation \ref{ex:double x}.
	\subsection{The Crank-Nicolson Approach}
	
	\noindent
	The approximations for the derivatives are used in
	the Crank-Nicolson approach, which is given by
	
	\begin{equation}
		\frac{u^{n+1}-u^{n}}{\Delta t} = \frac{1}{2}    \left(     F^{n+1} + F^n     \right),
	\end{equation}

	\noindent
	where the $F^{n+1}$ and $F^n$ terms are the right hand side of equation \ref{eq:bare Schrodinger}
	evaluated at the specified time steps. The superscript denotes the time step of the given variable, and $\Delta t$ is the length of the time step,
	i.e. the spacing between points on the time axis.\\ \\

	\noindent
	Rewriting equation \ref{eq:bare Schrodinger}
	according to the Crank-Nicolson approach we get 
	\begin{equation}\label{eq:algo}
		\begin{aligned}
		\frac{u^{n+1}_{i,j}-u^{n}_{i,j}}{\Delta t} &=  \frac{1}{2}\Big(\left(i\left(\frac{\partial^2 u_{i,j}}{\partial x^2} + \frac{\partial^2 u_{i,j}}{\partial y^2}\right) -i v_{i,j} u_{i,j}\right)^{n+1} \\
	    &+   \left(i(\frac{\partial^2 u_{i,j}}{\partial x^2} + \frac{\partial^2 u_{i,j}}{\partial y^2}) -i v_{i,j} u_{i,j}\right)^{n} \Big)\\
	    &\ \\
		&=  \frac{1}{2}\Big(    \frac{i}{h^2}\big(  u^{n+1}_{i+1,j}-2u^{n+1}_{i,j}+u^{n+1}_{i-1,j}   +u^{n+1}_{i,j+1}\\
		&-2u^{n+1}_{i,j}+u^{n+1}_{i,j-1}\big)-iv_{i,j}u^{n+1}_{i,j}\\
		&+ \frac{i}{h^2}\big(  u^{n}_{i+1,j}-2u^{n}_{i,j}+u^{n}_{i-1,j}   +u^{n}_{i,j+1}\\
		&-2u^{n}_{i,j}+u^{n}_{i,j-1}\big)-iv_{i,j}u^{n}_{i,j}      \Big)\\ 
		&\ \\
		&\Rightarrow \\ 
		&\ \\
		&u^{n+1}_{i,j} -r\Big(  u^{n+1}_{i+1,j}-2u^{n+1}_{i,j}+u^{n+1}_{i-1,j} \\
		&+u^{n+1}_{i,j+1}-2u^{n+1}_{i,j}+u^{n+1}_{i,j-1}\Big) +\frac{i\Delta t}{2}v_{i,j}u^{n+1}_{i,j}\\
		&\ \\
		&= u^n_{i,j}+ r\Big(  u^{n}_{i+1,j}-2u^{n}_{i,j}+u^{n}_{i-1,j}\\
		&+u^{n}_{i,j+1}-2u^{n}_{i,j}+u^{n}_{i,j-1}\Big)-\frac{i\Delta t}{2}v_{i,j}u^{n}_{i,j},
		\end{aligned}
	\end{equation}
	\noindent
	where we have used that $r = \frac{i\Delta t}{2h^2}$. The terms at time
	step $n+1$ are on the left hand side and terms at time step $n$ are on the right
	hand side.\\
	

	\subsection{The Simulated System}

	\noindent
	By using the Dirichlet boundary conditions in the $xy$ plane, i.e.
	\begin{itemize}
		\item $u(x=0, y, t) = 0$
		\item $u(x=1, y, t) = 0$
		\item $u(x, y=0, t) = 0$ 
		\item $u(x, y=1, t) = 0$,
	\end{itemize}
	we can write the result of equation \ref{eq:algo} as a matrix
	equation on the form 
	\begin{equation}
	Au^{n+1} = Bu^n,	
	\end{equation} 
	which can easily be solved when the initial condition of $u$ is known.\\ \\
	\noindent
	For the initial condition of the dimentionless wavefunction in two dimentions,
	i.e. $u^{0}$, we use an unnormalised Gaussian wave packet given by

	\begin{equation}\label{eq:wave packet}
		u(x,y,t=0) = e^{-\frac{(x-x_c)^2}{2 \sigma_x^2} - \frac{(y-y_c)^2}{2 \sigma_y^2} + i p_x (x-x_c) + i p_y (y-y_c)}.
	\end{equation}
	
	In equation \ref{eq:wave packet}, $x_c$ and $y_c$ are the central
	coordinates of the wave packet, $\sigma_x$ and $\sigma_y$ are the standard
	deviations in the $x$ and $y$ directions respectively and $p_x$ and $p_y$
	are the momenta of the wave packet.


	
	
	% ===========================================
	\section{Results}\label{sec:results}
	% ===========================================
	We now present the results for the open box, and the two-slit system, using the initial conditions in table \ref{tab:1}.
	\begin{table}[h!]
	\centering
	\caption{system parameters used for calculating the probability distribution for the no-slit and double slit.}
	\label{tab:1}
	\begin{tabular}{c | c | c} %Frode uses vertical lines, therefore, we can do it
	 Variable& No slit & Double slit\\
	\hline
	$h$                & $5\cross10^{-3}$                &\quad $5\cross10^{-3}$   \\
	$\Delta t$       & $2.5\cross10^{-5}$            &\quad $2.5\cross10^{-5}$  \\
	$T$                & $8\cross10^{-3}$               &\quad $8\cross10^{-3}$     \\
	$x_c$             & $2.5\cross10^{-1}$            &\quad $2.5\cross10^{-1}$  \\
	$\sigma_x$    & $5\cross10^{-2}$               &\quad $5\cross10^{-2}$     \\
	$p_x$             & $2\cross10^{2}$                &\quad $2\cross10^{2}$      \\
	$y_c$             & $5\cross10^{-1}$               &\quad $5\cross10^{-1}$     \\
	$\sigma_y$    & $5\cross10^{-2}$               &\quad $1\cross10^{-1}$     \\
	$p_y$             & $0$                                    &\quad $0$                         \\
	$v_0$             & $0$                                    &\quad $1\cross10^{10}$    \\
	\hline
	\end{tabular}
	\end{table}\\
	The resulting absolute error of these systems can, as functions of time, be seen in \ref{fig:prob7_error}.
	
	
	\begin{figure}[h!]
		\centering
		\includegraphics[scale=0.55]{figures/problem7_error.pdf}
		\caption{Absolute error versus time, using the system parameters in table \ref{tab:1}}
		\label{fig:prob7_error}
	\end{figure}
	We then find how the wave packet probability propagates through the two-slit system. First at time $t=0$.

	\begin{figure}[h!]
		\centering
		\includegraphics[trim={1cm 0cm 1cm 0cm},clip,width=0.49\textwidth]{figures/prob_plot_0.0.pdf}
		\caption{}
		\label{fig:prob_P0}
	\end{figure}

	\begin{figure}[h!]
		\centering
		\includegraphics[trim={1cm 0cm 1cm 0cm},clip,width=0.49\textwidth]{figures/prob_plot_0.001.pdf}
		\caption{}
		\label{fig:prob8_P1}
	\end{figure}

	\begin{figure}[h!]
		\centering
		\includegraphics[trim={1cm 0cm 1cm 0cm},clip,width=0.49\textwidth]{figures/prob_plot_0.002.pdf}
		\caption{}
		\label{fig:prob8_P2}
	\end{figure}

	\begin{figure}[h!]
		\centering
		\includegraphics[scale=0.55]{figures/real_plot_0.0.pdf}
		\caption{}
		\label{fig:prob8_Re0}
	\end{figure}

	\begin{figure}[h!]
		\centering
		\includegraphics[scale=0.55]{figures/real_plot_0.001.pdf}
		\caption{}
		\label{fig:prob8_Re1}
	\end{figure}

	\begin{figure}[h!]
		\centering
		\includegraphics[scale=0.55]{figures/real_plot_0.002.pdf}
		\caption{}
		\label{fig:prob8_Re2}
	\end{figure}

	\begin{figure}[h!]
		\centering
		\includegraphics[scale=0.55]{figures/im_plot_0.0.pdf}
		\caption{}
		\label{fig:prob8_Im0}
	\end{figure}
	
	\begin{figure}[h!]
		\centering
		\includegraphics[scale=0.55]{figures/im_plot_0.001.pdf}
		\caption{}
		\label{fig:prob8_Im1}
	\end{figure}
	
	\begin{figure}[h!]
		\centering
		\includegraphics[scale=0.55]{figures/im_plot_0.002.pdf}
		\caption{}
		\label{fig:prob8_Im2}
	\end{figure}

	\begin{figure}[h!]
		\centering
		\includegraphics[scale=0.55]{figures/single_slit_detection.pdf}
		\caption{}
		\label{fig:prob9_single}
	\end{figure}
	
	\begin{figure}[h!]
		\centering
		\includegraphics[scale=0.55]{figures/double_slit_detection.pdf}
		\caption{}
		\label{fig:prob9_double}
	\end{figure}
	
	\begin{figure}[h!]
		\centering
		\includegraphics[scale=0.55]{figures/triple_slit_detection.pdf}
		\caption{}
		\label{fig:prob9_triple}
	\end{figure}

	% ===========================================
	\section{Discussion}\label{sec:discussion}
	% ===========================================
		heihei
	% ===========================================
	\section{Conclusion}\label{sec:conclusion}
	% ===========================================
	
	In this paper we simulated a single non-relativistic particle obeying 
	Schrodinger's equation through three types of slit setup, single, double
	or triple slit. We observed that going through a slit most of the wave-
	function was reflected while a small part passed through. We saw the 
	interference pattern characteristic for a wave. \\
	
	We also represented the detection probability if we would have placed 
	a detector screen behind the wall at $x=0.8$. We found maxima and 
	minima due to constructive or destructive interference. \\
	
	All simulation passed the convergence check by conserving the total 
	probability value with a negligible error of order  $10^{-14}$. \\
	
	
	
	\onecolumngrid
	\section*{References}
	%\bibliographystyle{apalike}
	\bibliography{refs}
	
	
\end{document}
